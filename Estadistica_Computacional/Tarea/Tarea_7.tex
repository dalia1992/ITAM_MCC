\documentclass[]{article}
\usepackage{lmodern}
\usepackage{amssymb,amsmath}
\usepackage{ifxetex,ifluatex}
\usepackage{fixltx2e} % provides \textsubscript
\ifnum 0\ifxetex 1\fi\ifluatex 1\fi=0 % if pdftex
  \usepackage[T1]{fontenc}
  \usepackage[utf8]{inputenc}
\else % if luatex or xelatex
  \ifxetex
    \usepackage{mathspec}
  \else
    \usepackage{fontspec}
  \fi
  \defaultfontfeatures{Ligatures=TeX,Scale=MatchLowercase}
\fi
% use upquote if available, for straight quotes in verbatim environments
\IfFileExists{upquote.sty}{\usepackage{upquote}}{}
% use microtype if available
\IfFileExists{microtype.sty}{%
\usepackage{microtype}
\UseMicrotypeSet[protrusion]{basicmath} % disable protrusion for tt fonts
}{}
\usepackage[margin=1in]{geometry}
\usepackage{hyperref}
\hypersetup{unicode=true,
            pdftitle={Tarea 7: Simulación de modelos},
            pdfauthor={Dalia Camacho},
            pdfborder={0 0 0},
            breaklinks=true}
\urlstyle{same}  % don't use monospace font for urls
\usepackage{color}
\usepackage{fancyvrb}
\newcommand{\VerbBar}{|}
\newcommand{\VERB}{\Verb[commandchars=\\\{\}]}
\DefineVerbatimEnvironment{Highlighting}{Verbatim}{commandchars=\\\{\}}
% Add ',fontsize=\small' for more characters per line
\usepackage{framed}
\definecolor{shadecolor}{RGB}{248,248,248}
\newenvironment{Shaded}{\begin{snugshade}}{\end{snugshade}}
\newcommand{\KeywordTok}[1]{\textcolor[rgb]{0.13,0.29,0.53}{\textbf{#1}}}
\newcommand{\DataTypeTok}[1]{\textcolor[rgb]{0.13,0.29,0.53}{#1}}
\newcommand{\DecValTok}[1]{\textcolor[rgb]{0.00,0.00,0.81}{#1}}
\newcommand{\BaseNTok}[1]{\textcolor[rgb]{0.00,0.00,0.81}{#1}}
\newcommand{\FloatTok}[1]{\textcolor[rgb]{0.00,0.00,0.81}{#1}}
\newcommand{\ConstantTok}[1]{\textcolor[rgb]{0.00,0.00,0.00}{#1}}
\newcommand{\CharTok}[1]{\textcolor[rgb]{0.31,0.60,0.02}{#1}}
\newcommand{\SpecialCharTok}[1]{\textcolor[rgb]{0.00,0.00,0.00}{#1}}
\newcommand{\StringTok}[1]{\textcolor[rgb]{0.31,0.60,0.02}{#1}}
\newcommand{\VerbatimStringTok}[1]{\textcolor[rgb]{0.31,0.60,0.02}{#1}}
\newcommand{\SpecialStringTok}[1]{\textcolor[rgb]{0.31,0.60,0.02}{#1}}
\newcommand{\ImportTok}[1]{#1}
\newcommand{\CommentTok}[1]{\textcolor[rgb]{0.56,0.35,0.01}{\textit{#1}}}
\newcommand{\DocumentationTok}[1]{\textcolor[rgb]{0.56,0.35,0.01}{\textbf{\textit{#1}}}}
\newcommand{\AnnotationTok}[1]{\textcolor[rgb]{0.56,0.35,0.01}{\textbf{\textit{#1}}}}
\newcommand{\CommentVarTok}[1]{\textcolor[rgb]{0.56,0.35,0.01}{\textbf{\textit{#1}}}}
\newcommand{\OtherTok}[1]{\textcolor[rgb]{0.56,0.35,0.01}{#1}}
\newcommand{\FunctionTok}[1]{\textcolor[rgb]{0.00,0.00,0.00}{#1}}
\newcommand{\VariableTok}[1]{\textcolor[rgb]{0.00,0.00,0.00}{#1}}
\newcommand{\ControlFlowTok}[1]{\textcolor[rgb]{0.13,0.29,0.53}{\textbf{#1}}}
\newcommand{\OperatorTok}[1]{\textcolor[rgb]{0.81,0.36,0.00}{\textbf{#1}}}
\newcommand{\BuiltInTok}[1]{#1}
\newcommand{\ExtensionTok}[1]{#1}
\newcommand{\PreprocessorTok}[1]{\textcolor[rgb]{0.56,0.35,0.01}{\textit{#1}}}
\newcommand{\AttributeTok}[1]{\textcolor[rgb]{0.77,0.63,0.00}{#1}}
\newcommand{\RegionMarkerTok}[1]{#1}
\newcommand{\InformationTok}[1]{\textcolor[rgb]{0.56,0.35,0.01}{\textbf{\textit{#1}}}}
\newcommand{\WarningTok}[1]{\textcolor[rgb]{0.56,0.35,0.01}{\textbf{\textit{#1}}}}
\newcommand{\AlertTok}[1]{\textcolor[rgb]{0.94,0.16,0.16}{#1}}
\newcommand{\ErrorTok}[1]{\textcolor[rgb]{0.64,0.00,0.00}{\textbf{#1}}}
\newcommand{\NormalTok}[1]{#1}
\usepackage{graphicx,grffile}
\makeatletter
\def\maxwidth{\ifdim\Gin@nat@width>\linewidth\linewidth\else\Gin@nat@width\fi}
\def\maxheight{\ifdim\Gin@nat@height>\textheight\textheight\else\Gin@nat@height\fi}
\makeatother
% Scale images if necessary, so that they will not overflow the page
% margins by default, and it is still possible to overwrite the defaults
% using explicit options in \includegraphics[width, height, ...]{}
\setkeys{Gin}{width=\maxwidth,height=\maxheight,keepaspectratio}
\IfFileExists{parskip.sty}{%
\usepackage{parskip}
}{% else
\setlength{\parindent}{0pt}
\setlength{\parskip}{6pt plus 2pt minus 1pt}
}
\setlength{\emergencystretch}{3em}  % prevent overfull lines
\providecommand{\tightlist}{%
  \setlength{\itemsep}{0pt}\setlength{\parskip}{0pt}}
\setcounter{secnumdepth}{0}
% Redefines (sub)paragraphs to behave more like sections
\ifx\paragraph\undefined\else
\let\oldparagraph\paragraph
\renewcommand{\paragraph}[1]{\oldparagraph{#1}\mbox{}}
\fi
\ifx\subparagraph\undefined\else
\let\oldsubparagraph\subparagraph
\renewcommand{\subparagraph}[1]{\oldsubparagraph{#1}\mbox{}}
\fi

%%% Use protect on footnotes to avoid problems with footnotes in titles
\let\rmarkdownfootnote\footnote%
\def\footnote{\protect\rmarkdownfootnote}

%%% Change title format to be more compact
\usepackage{titling}

% Create subtitle command for use in maketitle
\newcommand{\subtitle}[1]{
  \posttitle{
    \begin{center}\large#1\end{center}
    }
}

\setlength{\droptitle}{-2em}

  \title{Tarea 7: Simulación de modelos}
    \pretitle{\vspace{\droptitle}\centering\huge}
  \posttitle{\par}
    \author{Dalia Camacho}
    \preauthor{\centering\large\emph}
  \postauthor{\par}
    \date{}
    \predate{}\postdate{}
  

\begin{document}
\maketitle

Supongamos que una compañía cambia la tecnología usada para producir una
cámara, un estudio estima que el ahorro en la producción es de 5 por
unidad con un error estándar de 4. Más aún, una proyección estima que el
tamaño del mercado (esto es, el número de cámaras que se venderá) es de
40,000 con un error estándar de 10,000. Suponiendo que las dos fuentes
de incertidumbre son independientes, usa simulación de variables
aleatorias normales para estimar el total de dinero que ahorrará la
compañía, calcula un intervalo de confianza.

\begin{Shaded}
\begin{Highlighting}[]
\KeywordTok{library}\NormalTok{(purrr)}
\KeywordTok{library}\NormalTok{(ggplot2)}
\end{Highlighting}
\end{Shaded}

Definimos los parámetros de venta de cámaras:

\begin{Shaded}
\begin{Highlighting}[]
\KeywordTok{set.seed}\NormalTok{(}\DecValTok{65646}\NormalTok{)}
\NormalTok{mu_n    <-}\StringTok{ }\DecValTok{40000}
\NormalTok{sigma_n <-}\StringTok{ }\KeywordTok{sqrt}\NormalTok{(}\DecValTok{10000}\NormalTok{)}
\NormalTok{Nsim    <-}\StringTok{ }\DecValTok{5000}
\end{Highlighting}
\end{Shaded}

Definimos los parámetros del ahorro:

\begin{Shaded}
\begin{Highlighting}[]
\NormalTok{mu_ah    <-}\StringTok{ }\DecValTok{5}
\NormalTok{sigma_ah <-}\StringTok{ }\KeywordTok{sqrt}\NormalTok{(}\DecValTok{4}\NormalTok{)}
\end{Highlighting}
\end{Shaded}

Para hacer una simulación del ahorro total se hace una simulación de la
venta total de cámaras, para cada cámara se simula el ahorro de fabricar
esa cámara en particular. Finalmente se suma el ahorro de todas las
cámaras.

\begin{Shaded}
\begin{Highlighting}[]
\NormalTok{Sim_ahorro <-}\StringTok{ }\ControlFlowTok{function}\NormalTok{()\{}
\NormalTok{  ncamaras <-}\StringTok{ }\KeywordTok{ceiling}\NormalTok{(}\KeywordTok{rnorm}\NormalTok{(}\DecValTok{1}\NormalTok{,mu_n,sigma_n))}
\NormalTok{  ahorro_i <-}\StringTok{ }\KeywordTok{rnorm}\NormalTok{(ncamaras,mu_ah, sigma_ah)}
\NormalTok{  thetasim <-}\StringTok{ }\KeywordTok{sum}\NormalTok{(ahorro_i)}
\NormalTok{  thetasim}
\NormalTok{\}}
\end{Highlighting}
\end{Shaded}

Corremos la simulación 5000 veces

\begin{Shaded}
\begin{Highlighting}[]
\NormalTok{ThetasSim <-}\StringTok{ }\KeywordTok{rerun}\NormalTok{(Nsim,}\KeywordTok{Sim_ahorro}\NormalTok{()) }\OperatorTok\StringTok{ }\KeywordTok{flatten_dbl}\NormalTok{()}

\KeywordTok{ggplot}\NormalTok{()}\OperatorTok{+}\KeywordTok{theme_bw}\NormalTok{()}\OperatorTok{+}
\StringTok{  }\KeywordTok{geom_histogram}\NormalTok{(}\KeywordTok{aes}\NormalTok{(ThetasSim),}\DataTypeTok{bins=}\DecValTok{30}\NormalTok{,}
                 \DataTypeTok{fill=}\StringTok{"pink"}\NormalTok{, }\DataTypeTok{col=}\StringTok{"pink"}\NormalTok{, }\DataTypeTok{alpha=}\FloatTok{0.8}\NormalTok{)}\OperatorTok{+}
\StringTok{  }\KeywordTok{xlab}\NormalTok{(}\StringTok{"Ahorro total"}\NormalTok{)}\OperatorTok{+}
\StringTok{  }\KeywordTok{ylab}\NormalTok{(}\StringTok{""}\NormalTok{)}
\end{Highlighting}
\end{Shaded}

\includegraphics{Tarea_7_files/figure-latex/unnamed-chunk-5-1.pdf}

Obtenemos el total estimado y su intervalo de confianza

\begin{Shaded}
\begin{Highlighting}[]
\NormalTok{theta  <-}\StringTok{ }\KeywordTok{mean}\NormalTok{(ThetasSim)}
\NormalTok{IC_low <-}\StringTok{ }\KeywordTok{quantile}\NormalTok{(ThetasSim, }\FloatTok{0.025}\NormalTok{)}
\NormalTok{IC_up  <-}\StringTok{ }\KeywordTok{quantile}\NormalTok{(ThetasSim, }\FloatTok{0.975}\NormalTok{)}
\end{Highlighting}
\end{Shaded}

El valor estimado del ahorro es de 1.9999\times 10\^{}\{5\}, con
intervalo de confianza (198731, 201246).


\end{document}
