\documentclass{article}
\usepackage[utf8]{inputenc}

\title{Arquitectura de computadoras: Ejercicio RTL}
\author{Dalia Camacho García Formentí }
\date{}

\usepackage{natbib}
\usepackage{graphicx}
\usepackage{float}
\usepackage{multicol}
\usepackage{amsmath}
\begin{document}

\maketitle
\section{Pregunta  1}
Diseñe una computadora capaz de realizar las siguientes instrucciones. Para las funciones control, basta con mencionar qué controla cada una y cuántas son. Debe incluirse diagrama a bloques de todos los componentes y sus interconexiones.

\begin{table}[H]
\centering 
\begin{tabular}{ccl}
Código & Mnemónico & Comentario\\
00 & LD INDIR & $A\leftarrow M[PTR]$.\\
01 & LDI PTR & $PTR\leftarrow dato $.\\
02 & INC PTR & $PTR\leftarrow PTR +1$.\\
03 & MOVR & $R\leftarrow A$.\\
04 & ADDR & $A\leftarrow A+R$.\\
05 & ADDI & $A\leftarrow A+dato$.\\
06 & SHL& $A\leftarrow shlA$.\\
07 & OR & $A\leftarrow A \lor R$.\\
\end{tabular}
\end{table}

\subsection*{Microoperaciones}
\begin{multicols}{2}
\textbf{FETCH}\\
$t_0:MAR \leftarrow PC$\\
$t_1: MBR \leftarrow M[MAR]$,  \\  $PC \leftarrow PC +1$\\
$t_2: IR \leftarrow MBR$\\\\

\noindent \textbf{LD INDIR}\\
$q_1t_3: MAR \leftarrow PTR$\\
$q_1t_4: MBR \leftarrow M[MAR]$\\
$q_1t_5: A\leftarrow MBR$,\\
$T\leftarrow 0$\\\\

\noindent \textbf{LDI PTR}\\
$q_2t3: MAR \leftarrow PC$\\
$q_2t_4: MBR\leftarrow M[MAR]$, \\ $PC \leftarrow PC +1$\\
$q_2t_5: PTR \leftarrow MBR$,\\
$T \leftarrow 0$\\\\

\noindent \textbf{INC PTR}\\
$q_3t_3: PTR \leftarrow PTR + 1$,\\
$T \leftarrow 0$\\\\

\noindent \textbf{MOVR}\\
$q_4t_3: R\leftarrow A$,\\
$T \leftarrow 0$\\\\

\noindent \textbf{ADDR}\\
$q_5t_3: A\leftarrow A+ R$, $T \leftarrow 0$\\\\
\noindent \textbf{ADDI}\\
$q_6t_3: MAR \leftarrow PC$\\
$q_6t_4: MBR\leftarrow M[MAR]$,\\  $PC \leftarrow PC +1$\\ 
$q_6t_5: A \leftarrow A+MBR$,\\
$T \leftarrow 0$\\\\

\noindent \textbf{SHL}\\
$q_7t_3: A \leftarrow shl(A)$,\\
$T \leftarrow 0$\\\\\\

\noindent \textbf{OR}\\
$q_8t_3: A \leftarrow A \lor R$,\\
$T \leftarrow 0$
\end{multicols}
\subsection*{Señales de control}
$MAR \leftarrow PC := X_1= t_0 + q_2t_3+q_6t_3$\\
$MAR \leftarrow PTR := X_2=q_1t_5$\\

\noindent $MBR\leftarrow M[MAR]:= X_3= t_1+q_1t_4+q_2t_4+q_6t_4$\\

\noindent $PC\leftarrow PC+1:= X_4=t_1+q_2t_4+q_6t_4$\\

\noindent $IR\leftarrow MBR:= X_5 = t_2$\\

\noindent $A\leftarrow MBR:= X_6= q_1t_5 + $\\
$A\leftarrow A+ R:= X_{7}=q_5t_3$\\
$A\leftarrow A +MBR:= X_{8}= q_6t_5$\\
$A\leftarrow shl(A):=X_{9}=q_7t_3$\\
$A\leftarrow A\lor R: X_{10}=q_8t_3$\\

\noindent $PTR \leftarrow MBR:= X_{11}=q_2t_5$\\
$PTR\leftarrow PTR+1: X_{12}=q_3t_3$\\

\noindent $R \leftarrow A: X_{13}=q_4t_3$\\

\noindent $T\leftarrow 0 := X_{14} = q_1t_5+q_2t_5+q_3t_3+q_4t_3+q_5t_3+q_6t_5+q_7t_3+q_8t_3$\\

\subsection{•}


\end{document}